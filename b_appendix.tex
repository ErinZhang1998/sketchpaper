% Appendixes should appear before the acknowledgment.
\subsection{Prepare Data for Training} \label{appendix:prepare_data}

\paragraph{Render Partial Sketches}
Go down the data-frame, for each entry, we have sketch category, sketch image index, and sketch part type. We need to locate how many parts are before the given part.  
For each entry, we can compute another quantity called strokes included. We can do this by: .
For example, entry $i$, it has strokes $i_{j_3}$ to $i_{j_5}$, meaning that it is composed of the 4th stroke to the 6th stroke. We need to render $i_{j_0}$ to $i_{j_2}$. 
Parameters: $CanvasSize$. Given images of rendered partial sketches: 
\begin{itemize}
    \item \texttt{render\_img(strokes, side = $CanvasSize$, original\_side = 256.)}
    \item \texttt{get\_strokes\_in\_parts(drawing\_raw, parts\_indices, canvas\_size = $CanvasSize$)} and then \texttt{render\_img(strokes, side = $CanvasSize$, original\_side = $CanvasSize$)}
\end{itemize}

\paragraph{Calculate Convex Hull for Each Semantic Part} 
If the part is a vertical line, meaning that $x_{min} = x_{max}$, then we sample 30 points on this vertical line and perturb 10 points with Gaussian noise (mean $0$ and std $0.05$) in the x-direction so that the part has length in the x-direction, then we can fit a convex hull using the \texttt{scipy.spatial} package. 


The primitives that we consider are: circle with radius 1, semi-circle with radius 1, semi-circle arc with radius 1, angled curve with side length $\sqrt{2}$, square with width 2. 